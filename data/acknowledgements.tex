% !TeX root = ../thuthesis-example.tex

\begin{acknowledgements}

清华求学三年,首先要感谢的是我的父母,是你们的多年供养让我能专心读书与科研;家庭作为我的支持,我倍感温暖和安心。还要感谢张凯老师,张凯老师是我研究生学习生涯的指路明灯:张老师不仅在专业方面对我有谆谆教诲,更是在人生选择和人格塑造方面对我有深刻影响。同时感谢数信院的老师们、华为中央媒体技术院的专家们,感谢你们在不同阶段给过我的指导和关心。

我还要感谢我一路同行的朋友和伙伴们,是你们让我的求学经历不再孤单。感谢同门的罗奇、陈子龙、王一博、邬豪杰、叶宇辰、卢汉卿、赵千淇和吴丹彤等同学,与你们的讨论和交流经常让我茅塞顿开,收获新发现与新思考;感谢郑钊宇、刘瑜平、罗赵彤和袁玉红等师兄师姐,你们的经验和指导使我在学习和研究中多了几分安心和从容。特别感谢丁桦同学,你在我三年的求学过程中给我面对困难的勇气,给我焦虑中的宽慰,给我痛苦中的关爱……你的存在让我在这个孤独的世界勇敢地走下去!

最后我还想感谢母校清华大学,感谢你对我人生观和世界观的塑造。回首求学的来时路,早年最令我印象深刻的一篇文章是《闻一多先生的说和做》:“他,是口的巨人。他,是行的高标”,这臧克家对闻一多先生的评价,也是我一生追求的人生目标。在清华,“行胜于言”的精神更为我的追求增加了一份使命感和荣誉感。在清华的日子很快就要结束了,但是我深感践行清华精神的旅程才刚刚开始,行胜于言将会是我日后时时自省的标尺。

无愧言行,无愧母校,吾当勉励!

\end{acknowledgements}
