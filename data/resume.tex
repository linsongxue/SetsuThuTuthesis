% !TeX root = ../thuthesis-example.tex

\begin{resume}

  \section*{个人简历}

  1999 年 04 月 21 日出生于山东淄博博山区。

  2017 年 9 月考入大连理工大学电信学部计算机科学与技术(日语强化)专业,2022 年 7 月本科毕业并获得工学学士学位。

  2022 年 9 月免试进入清华大学深圳国际研究生院攻读大数据技术与工程专业硕士至今。


  \section*{在学期间完成的相关学术成果}

  \subsection*{学术论文}

  \begin{achievements}
    % \item \textbf{Linsong Xue}, Luo Qi, Zhang Kai. APM-SLAM: Visual Localization for Fixed Routes with Tightly Coupled A Priori Map[J]. Journal of Intelligent and Connected Vehicles. (审稿中,一审Major revision)
    \item \textbf{Linsong Xue}, Kai Zhang, Guowei Zhu. Visual Localization with Prior Map. In press[C]. (已被World Congress on Intelligent Transport Systems 2025 Atlanta 录用, 清华-A类会议)
    \item Haojie Wu, \textbf{Linsong Xue}, Kai Zhang. SGAGS: Semantic-Guided Adaptive 3D Gaussian Splatting. In press[C]. (已被 the 5th International Conference on Image, Vision and Intelligent Systems录用)
  \end{achievements}


  \subsection*{专利}

  \begin{achievements}
    \item 张凯, 薛林松, 鹿昌义, 刘卫军. 一种间断GNSS信号下的车辆融合定位方法及相关设备: 中国, CN118112623A[P]. 2024-05-31. (发明专利,已公开)
    \item 张凯, 薛林松, 鹿昌义, 刘卫军. 一种基于NRTK和视觉信息的巡检车定位系统: 中国, CN119022939A[P]. 2024-11-26. (发明专利,已公开)
    \item 张凯, 罗奇, 李镇洋, 薛林松. 一种基于几何感知的特征匹配方法: 中国, CN2025104462718[P]. 2025-04-30. (发明专利,已受理)
  \end{achievements}

\end{resume}
