% !TeX root = ../thuthesis-example.tex

\begin{comments}
% \begin{comments}[name = {指导小组评语}]
% \begin{comments}[name = {Comments from Thesis Supervisor}]
% \begin{comments}[name = {Comments from Thesis Supervision Committee}]

论文选题聚焦于固定路线中的定位技术,在实际生活生产中具有广泛的应用前景。论文工作围绕视觉惯性定位方法展开,提出了一种基于先验地图和多传感器融合的定位系统,完成了地图构建、视觉惯性定位增强以及地图定位等多项工作,设计了一种完整的固定路线中视觉惯性定位系统,并通过多场景数据严谨论证了系统性能,包括精度和效率等方面。

论文的贡献包括:(1) 提出了基于 SfM 和 GNSS 融合的离线建图模块,有效得做到了建图成本与精度之间的平衡;(2) 提出了基于车身运动模式感知的伪观测视觉惯性里程计模块,有效得将车身的运动学先验融合到了视觉惯性里程计中;(3) 提出了基于概率地图的地图定位模块,有效得提升了定位精度和效率;(4) 通过多场景数据验证了系统效果,系统在精度和效率上均优于现有的视觉惯性定位系统以及地图定位系统。

论文显示该学生掌握了相关领域的知识,基础扎实,具有一定的独立研究能力;论文结构清晰,思路严谨,论证得当,写作规范,符合学术论文的要求,达到硕士学位论文要求。同意参加硕士论文答辩。

\end{comments}
