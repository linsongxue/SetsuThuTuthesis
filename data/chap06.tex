\chapter{总结与展望}

\section{论文总结}

固定路线中的视觉惯性定位方法是一项在自动驾驶、工业巡检等领域有着广泛需求的技术。但是当前的定位技术依赖于高成本的高精地图或者高精度传感器,这都严重制约了其在大规模应用中的推广。

为了解决这一问题,本文提出了一种基于视觉惯性的固定路线定位系统,该系统不依赖于人工标注的高精地图和昂贵的传感器,在建图阶段只需要图像和高精度的GNSS信息,在定位阶段只需要使用低成本的相机和IMU传感器,即可完成整个从建图到定位的流程。在建图到定位的整个流程中,本文需要解决的问题包括:(1) 如何利用图像和GNSS信息构建具有可接受精度的地图;(2) 如何融合IMU和相机信息进行较高精度的位姿估计;(3) 如何使用地图和位姿估计进行更精确的定位任务。针对这三个问题,本文提出了一个完整的固定路线中的视觉惯性定位系统,包括:

(1) 基于SfM和GNSS融合的离线建图模块。为了克服以往SLAM建图过程全局优化不足的问题,本文提出了一种基于SfM和GNSS融合的先验地图构建方法。首先通过图片和 GNSS 信息进行预处理,主要是筛选出候选关键帧以及进行语义分割等操作;紧接着纯图像首先进行 SfM 以获得在视觉建图世界坐标系下的相机位姿和地图点云;然后使用 GNSS 信息和 SfM 结果进行建图对齐,即实现视觉建图世界坐标系和全局坐标系的对齐;此后进行建图融合,以非线性优化的方式同时优化视觉重投影误差和 GNSS 提供的位置误差;最后输出分层次的的地图。

(2) 基于车身运动模式感知的伪观测视觉惯性里程计(PO-VIO)模块。为了解决以往通用视觉惯性里程计方法在车辆或轮式机器人上的速度估计不合理问题,本文设计了一种更符合车辆或轮式机器人运动状态的视觉惯性里程计:在通用VIO的基础上,本文引入了车辆在直行时车身横向和垂直速度的0伪观测,依据IMU的序列数据和基于深度学习的车身运动模式判别方法来检测车身状态变化,在特定状态下完成惯性-车体对齐并加入伪观测约束,从而获得更精确的位姿估计。

(3) 基于最大后验概率的地图定位模块。为了能够充分使用地图观测信息和视觉惯性里程计信息,本文设计了一种基于最大后验概率的,同步优化位姿与地图的紧耦合地图定位模块。整个模块首先通过比较图像观测与地图信息,利用粗粒度的图像特征和位置先验信息确定初步位置范围,再通过细粒度的像素特征匹配获得更精确的相机位姿。同时,通过转换矩阵初始化和初始位姿自适应选择来对齐视觉惯性里程计和地图观测,并验证地图定位结果的有效性。最终,通过将定位问题建模为最大后验概率估计,并以位姿图优化的形式求解最终定位结果,并通过不断更新转换参数来防止PO-VIO累积误差对整个系统的负面影响。

本文在公开数据集上对提出的方法进行了大量的实验验证,实验结果表明,本文提出的固定路线中的视觉惯性定位方法在定位精度和鲁棒性上均优于当前主流的视觉惯性定位方法,具有较好的实用性和推广性。

\section{未来展望}

在未来,本文考虑在以下几个方向进行进一步的研究:

(1) 当前的研究中使用地图作为重要的观测来源,但是每次使用时并不会对地图进行更新,这在长时间的使用中可能会导致地图的过时。因此,未来可以考虑在定位过程中对地图进行增量式更新,以提高地图的实时性和准确性。

(2) 本文在研究过程中发现动态物体对定位精度有较大的影响,因此未来可以考虑在定位过程中对动态物体进行检测和处理,以提高定位的鲁棒性。

(3) 基于深度学习的状态检测方法在本文的视觉惯性里程计部分发挥了重要作用,但是这种使用方法过于简单,未来考虑在更多的环节加入深度学习技术,以提高定位的精度和鲁棒性。