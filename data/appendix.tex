% !TeX root = ../thuthesis-example.tex

\chapter{补充内容}
% \label{appendix:quat}

\section{四元数与旋转矩阵转换}
\label{appendix:quat}
设单位四元数为
\begin{equation}
\symbf{q} = q_w + q_x i + q_y j + q_z k,
\end{equation}
满足归一化条件
\begin{equation}
q_w^2 + q_x^2 + q_y^2 + q_z^2 = 1.
\end{equation}
对应的旋转矩阵 \(\symbf{R}\) 可由下式给出:
\begin{equation}
  \symbf{R} = \begin{bmatrix}
1 - 2(q_y^2 + q_z^2) & 2(q_xq_y - q_zq_w) & 2(q_xq_z + q_yq_w) \\
2(q_xq_y + q_zq_w) & 1 - 2(q_x^2 + q_z^2) & 2(q_yq_z - q_xq_w) \\
2(q_xq_z - q_yq_w) & 2(q_yq_z + q_xq_w) & 1 - 2(q_x^2 + q_y^2)
\end{bmatrix}.
\end{equation}
给定旋转矩阵 \(\symbf{R} = [R_{ij}]\),可以通过下列方式提取四元数(其中一种常用的方法):
\begin{align}
  q_w &= \frac{1}{2}\sqrt{1 + R_{11} + R_{22} + R_{33}},\\
  q_x &= \frac{R_{32} - R_{23}}{4q_w},\\
  q_y &= \frac{R_{13} - R_{31}}{4q_w},\\
  q_z &= \frac{R_{21} - R_{12}}{4q_w}.
\end{align}

以上公式仅适用于单位四元数,且在实际应用中可能需要对数值稳定性进行进一步处理。


\section{光束法平差雅可比矩阵与优化}
\label{appendix:bundle_adjustment}
假设三维点 \( \mathbf{p} = (X, Y, Z)^T \) 在相机坐标系下的坐标为 \( \mathbf{p}_c \),其投影到图像平面上的像素坐标为 \( \mathbf{p} = (u, v)^T \)。相机的内参矩阵为 \( \mathbf{K} \),旋转矩阵为 \( \mathbf{R} \),平移向量为 \( \mathbf{t} \)。则有:
\begin{equation}
\mathbf{p}_c = \mathbf{R} \mathbf{p} + \mathbf{t}
\end{equation}
\begin{equation}
\mathbf{x} = \mathbf{K} \begin{bmatrix}
\frac{X_c}{Z_c} \\
\frac{Y_c}{Z_c} \\
1
\end{bmatrix}
\end{equation}
其中,\( \mathbf{p}_c = (X_c, Y_c, Z_c)^T \) 为三维点在相机坐标系下的坐标。
重投影误差 \( \mathbf{e} \) 定义为:
\begin{equation}
\mathbf{e} = \mathbf{x} - \mathbf{x}_{\text{obs}}
\end{equation}
其中,\( \mathbf{x}_{\text{obs}} \) 为实际观测到的像素坐标。
为了进行优化,需要计算重投影误差对旋转 \( \mathbf{R} \)、平移 \( \mathbf{t} \) 和地图点 \( \mathbf{P} \) 的雅可比矩阵。利用链式法则,雅可比矩阵可以分解为以下部分:
\begin{equation}
\frac{\partial \mathbf{e}}{\partial \mathbf{R}} = \frac{\partial \mathbf{e}}{\partial \mathbf{x}} \cdot \frac{\partial \mathbf{x}}{\partial \mathbf{p}_c} \cdot \frac{\partial \mathbf{p}_c}{\partial \mathbf{R}}
\end{equation}

\begin{equation}
\frac{\partial \mathbf{e}}{\partial \mathbf{t}} = \frac{\partial \mathbf{e}}{\partial \mathbf{x}} \cdot \frac{\partial \mathbf{x}}{\partial \mathbf{p}_c} \cdot \frac{\partial \mathbf{p}_c}{\partial \mathbf{t}}
\end{equation}

\begin{equation}
\frac{\partial \mathbf{e}}{\partial \mathbf{p}} = \frac{\partial \mathbf{e}}{\partial \mathbf{x}} \cdot \frac{\partial \mathbf{x}}{\partial \mathbf{p}_c} \cdot \frac{\partial \mathbf{p}_c}{\partial \mathbf{p}}
\end{equation}
其中,各部分的雅可比矩阵为:
\begin{equation}
\frac{\partial \mathbf{e}}{\partial \mathbf{x}} = \mathbf{I}_{2 \times 2}
\end{equation}

\begin{equation}
\frac{\partial \mathbf{p}}{\partial \mathbf{p}_c} = \mathbf{K} \cdot \begin{bmatrix}
\frac{1}{Z_c} & 0 & -\frac{X_c}{Z_c^2} \\
0 & \frac{1}{Z_c} & -\frac{Y_c}{Z_c^2} \\
0 & 0 & 0
\end{bmatrix}
\end{equation}

\begin{equation}
\frac{\partial \mathbf{p}_c}{\partial \mathbf{R}} = -\mathbf{R} \cdot \left[ \mathbf{p} \right]_\times
\end{equation}

\begin{equation}
\frac{\partial \mathbf{p}_c}{\partial \mathbf{t}} = \mathbf{I}_{3 \times 3}
\end{equation}

\begin{equation}
\frac{\partial \mathbf{p}_c}{\partial \mathbf{p}} = \mathbf{R}
\end{equation}
其中,\( \left[ \mathbf{p} \right]_\times \) 表示向量 \( \mathbf{P} \) 的反对称矩阵:
\begin{equation}
\left[ \mathbf{p} \right]_\times = \begin{bmatrix}
0 & -Z & Y \\
Z & 0 & -X \\
-Y & X & 0
\end{bmatrix}
\end{equation}
将上述各部分雅可比矩阵相乘,即可得到重投影误差对旋转、平移和地图点的雅可比矩阵。

高斯-牛顿法是一种用于求解非线性最小二乘问题的迭代算法。其目标是找到参数向量 \( \mathbf{x} \),使得残差的平方和最小化。具体而言,目标函数可以表示为:
\begin{equation}
F(\mathbf{x}) = \frac{1}{2} \sum_{i=1}^{m} r_i^2(\mathbf{x})
\end{equation}
其中,\( r_i(\mathbf{x}) \) 表示第 \( i \) 个残差函数,\( m \) 为残差的数量。
为了最小化 \( F(\mathbf{x}) \),我们需要找到使其梯度为零的 \( \mathbf{x} \)。首先,计算目标函数的梯度:
\begin{equation}
\nabla F(\mathbf{x}) = \sum_{i=1}^{m} r_i(\mathbf{x}) \nabla r_i(\mathbf{x})
\end{equation}
其中,\( \nabla r_i(\mathbf{x}) \) 是残差函数 \( r_i(\mathbf{x}) \) 关于 \( \mathbf{x} \) 的梯度。
接下来,对每个残差函数 \( r_i(\mathbf{x}) \) 进行一阶泰勒展开:
\begin{equation}
r_i(\mathbf{x} + \Delta \mathbf{x}) \approx r_i(\mathbf{x}) + \nabla r_i(\mathbf{x})^T \Delta \mathbf{x}
\end{equation}
将其代入目标函数的梯度表达式,得到:
\begin{equation}
\nabla F(\mathbf{x} + \Delta \mathbf{x}) \approx \sum_{i=1}^{m} \left[ r_i(\mathbf{x}) + \nabla r_i(\mathbf{x})^T \Delta \mathbf{x} \right] \nabla r_i(\mathbf{x})
\end{equation}
为了使梯度为零,需要解以下线性方程组:
\begin{equation}
\mathbf{J}(\mathbf{x})^T \mathbf{r}(\mathbf{x}) + \mathbf{J}(\mathbf{x})^T \mathbf{J}(\mathbf{x}) \Delta \mathbf{x} = 0
\end{equation}
其中,\( \mathbf{J}(\mathbf{x}) \) 是残差函数的雅可比矩阵,\( \mathbf{r}(\mathbf{x}) \) 是所有残差的向量表示。
解上述方程组,得到参数更新量:
\begin{equation}
\Delta \mathbf{x} = -\left[ \mathbf{J}(\mathbf{x})^T \mathbf{J}(\mathbf{x}) \right]^{-1} \mathbf{J}(\mathbf{x})^T \mathbf{r}(\mathbf{x})
\end{equation}
然后,更新参数:
\begin{equation}
\mathbf{x} \leftarrow \mathbf{x} + \Delta \mathbf{x}
\end{equation}
重复上述步骤,直到满足收敛条件。


\section{罗德里格斯旋转公式}
\label{appendix:rodrigues}
罗德里格斯旋转公式给定一个旋转矩阵 \( \symbf{R} \),可以将其转换为旋转向量 \( \boldsymbol{\theta} = \theta \symbf{u} \),其中 \( \theta \) 是旋转角度,\( \symbf{u} \) 是旋转轴的单位向量。旋转角度 \( \theta \)的计算方式如下:

\begin{equation}
\theta = \cos^{-1} \left( \frac{\text{trace}(\symbf{R}) - 1}{2} \right)
\end{equation}
其中,\( \text{trace}(\symbf{R}) \) 表示矩阵 \( \symbf{R} \) 的迹,即对角线元素之和。旋转轴 \( \symbf{u} \)是一个单位向量,其计算方式为:
\begin{equation}
\symbf{u} = \frac{1}{2 \sin \theta} \begin{bmatrix}
\symbf{R}_{32} - \symbf{R}_{23} \\
\symbf{R}_{13} - \symbf{R}_{31} \\
\symbf{R}_{21} - \symbf{R}_{12}
\end{bmatrix}
\end{equation}
其中,\( \symbf{R}_{ij} \) 表示矩阵 \( \symbf{R} \) 的第 \( i \) 行第 \( j \) 列元素。


